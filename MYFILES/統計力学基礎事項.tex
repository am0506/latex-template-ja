\documentclass[11pt,a4paper]{jarticle}
\usepackage{graphicx}
\usepackage[dvipdfmx]{color}
\usepackage{amsmath}
\usepackage{amsthm}
\usepackage{mathtools}
\usepackage{amssymb}
\usepackage{bm}
\usepackage{ascmac}
\usepackage{braket}
\usepackage[top=2.5cm, bottom=2.5cm, left=2cm, right=2cm]{geometry}
\usepackage{tikz}
\usepackage{xcolor}
\usepackage{here}
\usetikzlibrary{intersections,calc,arrows.meta}
\title{統計力学基本事項}
\author{}
\date{\today}
\begin{document}
\maketitle
\begin{enumerate}
    \item ミクロカノニカル分布
    \begin{enumerate}
        \item ミクロカノニカル分布の仮定について説明し, ミクロカノニカル分布を導出する.
        \item ミクロカノニカル分布を用いて古典理想気体のエントロピー, エネルギーを求める.
    \end{enumerate}
    \item カノニカル分布
    \begin{enumerate}
        \item ミクロカノニカル分布を用いてセットアップを行い, カノニカル分布を導出する.
    \end{enumerate}
\end{enumerate}
\end{document}