\documentclass[11pt,a4paper]{jarticle}
\usepackage{graphicx}
\usepackage[dvipdfmx]{color}
\usepackage{amsmath}
\usepackage{amsthm}
\usepackage{mathtools}
\usepackage{amssymb}
\usepackage{bm}
\usepackage{ascmac}
\usepackage{braket}
\usepackage[top=2.5cm, bottom=2.5cm, left=2cm, right=2cm]{geometry}
\usepackage{tikz}
\usepackage{xcolor}
\usepackage{here}
\usetikzlibrary{intersections,calc,arrows.meta}
\title{統計力学基本事項}
\author{}
\date{\today}
\begin{document}
\maketitle
\begin{enumerate}
    \item ミクロカノニカル分布
    \begin{enumerate}
        \item ミクロカノニカル分布の仮定について説明し, ミクロカノニカル分布を導出する.
        \item ミクロカノニカル分布を用いて古典理想気体のエントロピー, エネルギーを求める.
    \end{enumerate}
    \item カノニカル分布
    \begin{enumerate}
        \item ミクロカノニカル分布を用いてセットアップを行い, カノニカル分布を導出する.
        \item カノニカル分布を用いて理想気体に対する物理量を求め, 熱力学との対応について説明する.
        \item 2準位系について基本的な物理量を求める.
        \item 軌道角運動量$J$を持つスピン系に対して基本的な物理量を求める.
    \end{enumerate}
    \item グランドカノニカル分布
    \begin{enumerate}
        \item ミクロカノニカル分布を用いてセットアップを行い, グランドカノニカル分布を導出する.
        \item グランドカノニカル分布を用いて, 理想気体の物理量を求める.
        \item ラングミュア―吸着について説明する.
    \end{enumerate}
    \item Bose粒子とFermi粒子
    \begin{enumerate}
        \item Bose統計とFermi統計に従って, Bose分布関数とFermi分布関数を導出する.
        \item Fermi粒子について
        \begin{enumerate}
            \item 理想Fermi気体の比熱について, 温度領域を場合分けして求めよ.
        \end{enumerate}
    \end{enumerate}
\end{enumerate}
\end{document}