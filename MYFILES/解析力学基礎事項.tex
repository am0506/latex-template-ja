\documentclass[11pt,a4paper]{jarticle}
\usepackage{graphicx}
\usepackage[dvipdfmx]{color}
\usepackage{amsmath}
\usepackage{amsthm}
\usepackage{mathtools}
\usepackage{amssymb}
\usepackage{bm}
\usepackage{ascmac}
\usepackage{braket}
\usepackage[top=2.5cm, bottom=2.5cm, left=2cm, right=2cm]{geometry}
\usepackage{tikz}
\usepackage{xcolor}
\usepackage{here}
\usetikzlibrary{intersections,calc,arrows.meta}
\title{}
\author{解析力学基本事項}
\date{}
\begin{document}
\maketitle
\item Euler Lagrange方程式
\begin{enumerate}
    \item Hamiltonの最小作用の原理からEuler Lagrange方程式を導出する.
\end{enumerate}
\item 連成振動
\begin{enumerate}
    \item 連成振動について状況を設定し, 運動方程式を解くことで説明せよ.
\end{enumerate}
\item 正準変換
\end{document}