\documentclass[11pt,a4paper]{jarticle}
\usepackage{graphicx}
\usepackage[dvipdfmx]{color}
\usepackage{amsmath}
\usepackage{amsthm}
\usepackage{mathtools}
\usepackage{amssymb}
\usepackage{bm}
\usepackage{ascmac}
\usepackage{braket}
\usepackage[top=3.5cm, bottom=3.5cm, left=2.5cm, right=2.5cm]{geometry}
\usepackage{tikz}
\usepackage{xcolor}
\usepackage{here}
\usetikzlibrary{intersections,calc,arrows.meta}
\title{量子力学基本事項}
\author{}
\date{\today}
\begin{document}
\maketitle
\begin{enumerate}
    \item シュレディンガー方程式とそれらを解く
    \begin{enumerate}
        \item フーリエ変換を用いてシュレディンガー方程式を導出する.
        \item 時間に依存しないシュレディンガー方程式を導出し, 定常状態について説明する.
        \item 1次元波動関数について一般的に言えることを説明する.
        \item よくあるポテンシャル(有限井戸型ポテンシャル, ステップポテンシャル, 無限井戸型ポテンシャル, デルタ関数型ポテンシャル)について解く.
        \item ガウス型波束が初期条件として与えられたときの任意の時間の波動関数を求める.
        \item 調和振動子ポテンシャルについて解く.
    \end{enumerate}
    \item 調和振動子の代数的解法
    \begin{enumerate}
        \item 調和振動子ポテンシャルについてHamiltonianを書き換えて解け.
    \end{enumerate}
    \item スピン
    \begin{enumerate}
        \item 一般の角運動量の交換関係から, 位置に関する角運動量の固有関数を固有値を求めよ.
        \item スピンの合成について説明し, 2つの$s=\frac{1}{2}$を合成したときの固有関数を求めよ.
    \end{enumerate}
    \item 摂動論
    \begin{enumerate}
        \item 1次,2次の摂動エネルギーと関数を導出する.
    \end{enumerate}
\end{enumerate}
\end{document}