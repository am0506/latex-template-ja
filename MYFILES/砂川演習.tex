\documentclass[11pt,a4paper]{jarticle}
\usepackage{graphicx}
\usepackage[dvipdfmx]{color}
\usepackage{amsmath}
\usepackage{amsthm}
\usepackage{mathtools}
\usepackage{amssymb}
\usepackage{bm}
\usepackage{ascmac}
\usepackage{braket}
\usepackage[top=3.5cm, bottom=3.5cm, left=2.5cm, right=2.5cm]{geometry}
\usepackage{tikz}
\usepackage{xcolor}
\usepackage{here}
\usetikzlibrary{intersections,calc,arrows.meta}
\title{}
\author{}
\date{}
\begin{document}
\maketitle
\begin{enumerate}
    \item 静電場
    \begin{enumerate}
        \item 球対称の電荷分布を持つ系の電場について議論せよ. 特に一様に分布する場合と表面にのみ分布する場合について考えよ.
        \item 双極子モーメントが十分遠くで作る電場を求めよ.
        \item 電位差が与えられているときの平板コンデンサーの作る電場と表面に作る電荷を求めよ.
        \item 導体球とその外側に双極子モーメント$\bm{p}$(ただし, 向きは導体球の外側を向いているとする)があるとき, 導体球表面に与えられる表面電荷を求めよ.
        \item 1次元電荷分布を仮想的に考え, その密度の変化を考えることで, 電子はどのような運動をするか論じよ.
    \end{enumerate}
\end{enumerate}
\end{document}