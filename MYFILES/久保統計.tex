\documentclass[11pt,a4paper]{jarticle}
\usepackage{graphicx}
\usepackage[dvipdfmx]{color}
\usepackage{amsmath}
\usepackage{amsthm}
\usepackage{mathtools}
\usepackage{amssymb}
\usepackage{bm}
\usepackage{ascmac}
\usepackage{braket}
\usepackage[top=3.5cm, bottom=3.5cm, left=2.5cm, right=2.5cm]{geometry}
\usepackage{tikz}
\usepackage{xcolor}
\usepackage{here}
\usetikzlibrary{intersections,calc,arrows.meta}
\title{歩く久保統計ゼミ}
\author{}
\date{}
\begin{document}
\maketitle
\begin{enumerate}
    \item 5/30
    \begin{enumerate}
        \item Thomsonの原理からClausiusの原理を導け. 逆も導け.
        \item 一様な重力場日ある気体の圧力$p$は温度が一様であれば
        \begin{equation}
            p(z)=p(0)\exp(-\frac{mgz}{kT})
        \end{equation}
        であることを示せ.
        \item 一定の大きさ$\mu$の電気2重極子モーメントを持つ2原子分子$N$個からなる理想気体の電気分極が
        \begin{equation*}
            P=\frac{N}{V}\mu(\coth(\frac{\mu E}{kT}-\frac{kT}{\mu E}))
        \end{equation*}
        となることを示せ.
        \item
    \end{enumerate}
\end{enumerate}
\end{document}