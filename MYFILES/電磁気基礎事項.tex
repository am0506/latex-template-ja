\documentclass[11pt,a4paper]{jarticle}
\usepackage{graphicx}
\usepackage[dvipdfmx]{color}
\usepackage{amsmath}
\usepackage{amsthm}
\usepackage{mathtools}
\usepackage{amssymb}
\usepackage{bm}
\usepackage{ascmac}
\usepackage{braket}
\usepackage[top=2.5cm, bottom=2.5cm, left=2cm, right=2cm]{geometry}
\usepackage{tikz}
\usepackage{xcolor}
\usepackage{here}
\usetikzlibrary{intersections,calc,arrows.meta}
\title{電磁気基本事項}
\author{}
\date{}
\begin{document}
\maketitle
\begin{enumerate}
    \item Maxwell方程式
    \begin{enumerate}
        \item Maxwell方程式を実験事実から導く.
        \item Maxwell方程式から波動方程式を導く
    \end{enumerate}
    \item 静電場
    \begin{enumerate}
        \item ガウスの法則を導く.
        \item 一般に電荷分布が与えられたときの電場の分布はどのように決定されるか?
        \item 電荷分布および導体が与えられたときの電場について説明せよ.
        \item 誘電体中に電荷があるときの静電場について説明せよ.
        \item 静電場の持つエネルギーおよび応力について説明せよ.
    \end{enumerate}
    \item 定常電流
    \begin{enumerate}
        \item 電荷保存則から定常電流の保存則を導け.
        \item オームの法則が成立するとして, キルヒホッフの法則および定常電流の空間分布を一般的に求めよ.
    \end{enumerate}
    \item 静磁場
    \begin{enumerate}
        \item ビオサバールの法則を導く.
    \end{enumerate}
    \item 電磁波
\end{enumerate}
\end{document}