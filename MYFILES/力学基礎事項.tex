\documentclass[11pt,a4paper]{jarticle}
\usepackage{graphicx}
\usepackage[dvipdfmx]{color}
\usepackage{amsmath}
\usepackage{amsthm}
\usepackage{mathtools}
\usepackage{amssymb}
\usepackage{bm}
\usepackage{ascmac}
\usepackage{braket}
\usepackage[top=3.5cm, bottom=3.5cm, left=2.5cm, right=2.5cm]{geometry}
\usepackage{tikz}
\usepackage{xcolor}
\usepackage{here}
\usetikzlibrary{intersections,calc,arrows.meta}
\title{力学基本事項}
\author{}
\date{\today}
\begin{document}
\maketitle
\begin{enumerate}
    \item 軌道学
    \begin{enumerate}
        \item 3次元直交座標系と極座標の変換を行い, 速度および加速度を極座標で表す.
        \item 同様のことを円柱座標で行う.
    \end{enumerate}
    \item 剛体の力学
    \begin{enumerate}
        \item 慣性モーメントの定義を述べる。また, 平行軸の定理および直交定理を導く。
    \end{enumerate}
\end{enumerate}

\end{document}